\nonstopmode{}
\documentclass[a4paper]{book}
\usepackage[times,inconsolata,hyper]{Rd}
\usepackage{makeidx}
\usepackage[utf8]{inputenc} % @SET ENCODING@
% \usepackage{graphicx} % @USE GRAPHICX@
\makeindex{}
\begin{document}
\chapter*{}
\begin{center}
{\textbf{\huge Package `silp'}}
\par\bigskip{\large \today}
\end{center}
\inputencoding{utf8}
\ifthenelse{\boolean{Rd@use@hyper}}{\hypersetup{pdftitle = {silp: Conditional Process Analysis (CPA) via SEM Approach}}}{}
\ifthenelse{\boolean{Rd@use@hyper}}{\hypersetup{pdfauthor = {Yi-Hsuan Tseng; Po-Hsien Huang}}}{}
\begin{description}
\raggedright{}
\item[Title]\AsIs{Conditional Process Analysis (CPA) via SEM Approach}
\item[Version]\AsIs{1.0.0}
\item[Description]\AsIs{The ``silp'' package utilizes the Reliability-Adjusted Product Indicator (RAPI) method to 
estimate effects among latent variables, thus allowing for more precise definition and analysis of 
mediation and moderation models. Our simulation studies reveal that while ``silp'' may exhibit 
instability with smaller sample sizes and lower reliability scores (e.g., 𝑁 = 100, 𝜔 = 0.7), 
implementing nearest positive definite matrix correction and bootstrap confidence interval 
estimation can significantly ameliorate this volatility. When these adjustments are applied, 
``silp'' achieves estimations akin in quality to those derived from LMS. In conclusion, the ``silp'' 
package is a valuable tool for researchers seeking to explore complex relational structures between 
variables without resorting to commercial software.}
\item[License]\AsIs{MIT + file LICENSE}
\item[Encoding]\AsIs{UTF-8}
\item[Roxygen]\AsIs{list(markdown = TRUE)}
\item[RoxygenNote]\AsIs{7.3.2}
\item[Imports]\AsIs{Matrix,
methods,
lavaan,
MASS,
purrr,
semTools,
stats,
stringr}
\item[NeedsCompilation]\AsIs{no}
\item[Maintainer]\AsIs{Yi-Hsuan Tseng }\email{r12227115@g.ntu.edu.tw}\AsIs{}
\item[URL]\AsIs{}\url{https://github.com/TomBJJJ/silp}\AsIs{}
\item[BugReports]\AsIs{}\url{https://github.com/TomBJJJ/silp/issues}\AsIs{}
\end{description}
\Rdcontents{\R{} topics documented:}
\inputencoding{utf8}
\HeaderA{generate\_data}{generate\_data}{generate.Rul.data}
%
\begin{Description}
Generates data based on the simulation settings provided by Cheung et al. (2021).
Note that the reliability used here is \code{omega}.
\end{Description}
%
\begin{Usage}
\begin{verbatim}
generate_data(
  n_obs = 100,
  corr = 0.3,
  effect = 0.42,
  ld = c(1, 1, 1, 1),
  alp = 0.9,
  effect_x = 0.4,
  effect_z = 0.2
)
\end{verbatim}
\end{Usage}
%
\begin{Arguments}
\begin{ldescription}
\item[\code{n\_obs}] Integer. The number of observations.

\item[\code{corr}] Numeric. The correlation of the latent variables.

\item[\code{effect}] Numeric. The effect of the moderator.

\item[\code{ld}] Numeric. The factor loading of the latent variable to its indicators.

\item[\code{alp}] Numeric. The reliability of the latent variable.

\item[\code{effect\_x}] Numeric. The direct effect of x.

\item[\code{effect\_z}] Numeric. The direct effect of z.
\end{ldescription}
\end{Arguments}
%
\begin{Value}
A dataset simulated from the argument settings.
\end{Value}
%
\begin{Examples}
\begin{ExampleCode}
n_obs = 100
corr = 0.1
effect = 0.12
ld = c(1,1,1,1)
alp = 0.9
generate_data(n_obs, corr, effect, ld, alp)
\end{ExampleCode}
\end{Examples}
\inputencoding{utf8}
\HeaderA{resilp}{resilp}{resilp}
%
\begin{Description}
An extended function from \code{silp}, applying the bootstrap method to obtain standard error estimation.
Note: When using \code{silp} with the nearest positive definite matrix (npd = TRUE), this function should be used to obtain
reliable inference.
\end{Description}
%
\begin{Usage}
\begin{verbatim}
resilp(fit, R = 2000, progress = T)
\end{verbatim}
\end{Usage}
%
\begin{Arguments}
\begin{ldescription}
\item[\code{fit}] A result object from \code{silp}.

\item[\code{R}] Integer. The number of bootstrap samples. Default is 2000.

\item[\code{progress}] Logical. Whether to display a progress bar. Default is \code{FALSE}.
\end{ldescription}
\end{Arguments}
%
\begin{Value}
An object of class "Silp".
\end{Value}
%
\begin{Examples}
\begin{ExampleCode}
n_obs = 100
corr = 0.1
effect = 0.12
ld = c(1,1,1,1)
alp = 0.9
data = generate_data(n_obs, corr, effect, ld, alp)
model = "
  fy =~ y1 + y2 + y3 + y4
  fx =~ x1 + x2 + x3 + x4
  fz =~ z1 + z2 + z3 + z4
  fy ~  fx + fz + fx:fz
"
fit = silp(model, data)
resilp(fit, R = 10)
\end{ExampleCode}
\end{Examples}
\inputencoding{utf8}
\HeaderA{silp}{silp}{silp}
%
\begin{Description}
This function extends the \code{lavaan} function, allowing users to define moderation effects using the symbol ":".
The RAPI method is used to estimate moderation effects.
\end{Description}
%
\begin{Usage}
\begin{verbatim}
silp(model, data, center = "double", tau.eq = F, npd = F, ...)
\end{verbatim}
\end{Usage}
%
\begin{Arguments}
\begin{ldescription}
\item[\code{model}] A \code{lavaan} syntax model with extension. The notation ":" implies interaction between two variables (see Example).

\item[\code{data}] The dataset for \code{lavaan} SEM.

\item[\code{center}] Character. Whether single or double mean centering is used for the product indicator. Default is "double".

\item[\code{tau.eq}] Logical. Specifies the type of reliability used to estimate error variance. If \code{TRUE}, Cronbach's alpha reliability is used.
If \code{FALSE}, omega reliability is used. Default is \code{FALSE}.

\item[\code{npd}] Logical. Specifies the type of input used in \code{lavaan} SEM. Default is \code{FALSE} for raw data or \code{TRUE} for a covariance matrix.
Applying a covariance matrix can resolve problems of a non-positive definite covariance matrix.
If \code{TRUE}, \code{resilp} should be used to obtain reliable inference.

\item[\code{...}] Other parameters passed to the \code{lavaan} SEM function.\#'
\end{ldescription}
\end{Arguments}
%
\begin{Value}
An "Silp" class object.
\end{Value}
%
\begin{Examples}
\begin{ExampleCode}
n_obs = 100
corr = 0.1
effect = 0.12
ld = c(1,1,1,1)
alp = 0.9
data = generate_data(n_obs, corr, effect, ld, alp)
model = "
  fy =~ y1 + y2 + y3 + y4
  fx =~ x1 + x2 + x3 + x4
  fz =~ z1 + z2 + z3 + z4
  fy ~  fx + fz + fx:fz
"
silp(model, data)
\end{ExampleCode}
\end{Examples}
\inputencoding{utf8}
\HeaderA{Silp-class}{Define silp class}{Silp.Rdash.class}
%
\begin{Description}
Define silp class
\end{Description}
%
\begin{Section}{Slots}

\begin{description}

\item[\code{raw\_model}] The user-specified \code{lavaan} syntax model.

\item[\code{rapi\_model}] The revised model with the RAPI method.

\item[\code{time}] The operation time for \code{silp} (in seconds).

\item[\code{npd}] Logical. Whether the nearest positive definite matrix is used.

\item[\code{raw\_data}] The input data.

\item[\code{fa}] An object of class \code{lavaan} representing the CFA result.

\item[\code{reliability}] The reliability index.

\item[\code{composite\_data}] The composite data for RAPI.

\item[\code{pa}] The result of \code{silp}.

\item[\code{boot}] The results of \code{resilp} from R bootstrap samples.

\item[\code{origine}] The original \code{silp} estimation.

\item[\code{time\_resilp}] The operation time for \code{resilp} (in seconds).

\end{description}
\end{Section}
\inputencoding{utf8}
\HeaderA{summary-methods}{ Methods for Class \code{Silp} in Package \pkg{silp}}{summary.Rdash.methods}
\aliasA{summary,Silp-method}{summary-methods}{summary,Silp.Rdash.method}
%
\begin{Description}
Summary Methods for Class \code{Silp} in Package \pkg{silp}.
\end{Description}
%
\begin{Section}{Methods}
\begin{description}

\item[\code{signature(object = "Silp", method = "Bootstrap")}] 
Returns the summary result of `silp` or `resilp`. This method is for `resilp` only. If \code{method = "Bootstrap"}, the percentile bootstrap result is presented. If \code{method = "BC\_b"}, the bias-corrected bootstrap result is presented. 


\end{description}
\end{Section}
\printindex{}
\end{document}
